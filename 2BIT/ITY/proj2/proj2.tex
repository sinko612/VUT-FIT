\documentclass[a4paper, twocolumn, 11pt]{article}
\usepackage[text={18.2cm, 25.2cm},left=1.4cm, top=2.3cm]{geometry}
\usepackage[czech]{babel}
\usepackage[IL2]{fontenc}
\usepackage[utf8]{inputenc}
\usepackage{times}
\usepackage{amsthm, amssymb, amsmath}

\newtheorem{definicia}{Definice}
\newtheorem{veta}{Věta}

\begin{document}
    \begin{titlepage}
        \begin{center}
           \textsc {\Huge{Vysoké učení technické v~Brně}\\[0,5em]
                     \huge{Fakulta informačních technologií}\\[0,4em]}
            \vspace{\stretch{0.382}}
            {\LARGE Typografie a~publikování\,--\,2. projekt\\[0,5em]
        	      Sazba dokumentů a~matematických výrazů\\[0,4em]}
            \vspace{\stretch{0.618}}
        \end{center}
        {\Large 2023 \hfill Simona Jánošíková (xjanos19)}
    \end{titlepage}

\section*{Úvod}
V~této úloze si vyzkoušíme sazbu titulní strany, matematických vzorců, prostředí a~dalších textových struktur obvyklých pro technicky zaměřené texty\,--\,například Definice~\ref{definicia1} nebo rovnice~\eqref{rov3} na straně~\pageref{rov3}. Pro vytvoření těchto odkazů používáme kombinace příkazů \verb|\label|, \verb|\ref|, \verb|\eqref| a~\verb|\pageref|. Před odkazy patří nezlomitelná mezera. Pro zvýrazňování textu jsou zde několikrát použity příkazy \verb|\verb| a~\verb|\emph|. 

Na titulní straně je použito prostředí \verb|titlepage| a~sázení nadpisu podle optického středu s~využitím \emph{přesného} zlatého řezu. Tento postup byl probírán na přednášce. Dále jsou na titulní straně použity čtyři různé velikosti písma a~mezi dvojicemi řádků textu je použito odřádkování se zadanou relativní velikostí 0,5 em a~0,4 em\footnote{Nezapomeňte použít správný typ mezery mezi číslem a~jednotkou.}.

\section{Matematický text}
V~této sekci se podíváme na sázení matematických symbolů a~výrazů v~plynulém textu pomocí prostředí \verb|math|. Definice a~věty sázíme pomocí příkazu \verb|\newtheorem| s~využitím balíku \verb|amsthm|. Někdy je vhodné použít konstrukci \verb|${}$| nebo \verb|\mbox{}|, která říká, že (matematický) text nemá být zalomen. 

\begin{definicia}\label{definicia1}
    \emph{Zásobníkový automat} (ZA) je definován jako sedmice tvaru $A = (Q,\Sigma,\Gamma,\delta,q_0,Z_0,F)$ , kde: 
    \begin{itemize}
        \item $Q$ je konečná množina \emph{vnitřních (řídicích) stavů,}
        \item $\Sigma$ je konečná \emph{vstupní abeceda,}
        \item $\Gamma$ je konečná \emph{zásobníková abeceda,}
        \item $\delta$ je \emph{přechodová funkce} $Q\times(\Sigma\cup\{\epsilon\})\times\Gamma\rightarrow 2^{Q\times\Gamma^{*}}$,
        \item $q_0 \in Q$ je \emph{počáteční stav}, $Z_0 \in \Gamma$ je \emph{startovací symbol zásobníku} a~$F \subseteq Q$ je množina \emph{koncových stavů}.
    \end{itemize}
\end{definicia}


Nechť $P = (Q,\Sigma,\Gamma,\delta,q_0,Z_0,F)$ je ZA. \emph{Konfigurací} nazveme trojici $(q,w,\alpha)\in Q \times\Sigma^{*}\times\Gamma^{*}$, kde \emph{q} je aktuální stav vnitřního řízení, $w$ je dosud nezpracovaná část vstupního řetězce a~$\alpha = Z_{i_1}Z_{i_2} \dots Z_{i_k}$ je obsah zásobníku.

\subsection{Podsekce obsahující definici a~větu}
\begin{definicia}
    \emph{Řetězec $w$ nad abecedou $\Sigma$ je přijat ZA} $A$~jestliže $(q_0,w,Z_0) \overset{*}{\underset{A}\vdash} (q_F,\epsilon,\gamma)$ pro nějaké $\gamma \in \Gamma^{*}$ a $q_F \in F$. 
    Množina $L(A)=\{w \mid w$ je přijat ZA $A\} \subseteq \Sigma^{*}$ \emph{je jazyk přijímaný ZA}~$A$.
\end{definicia}
\begin{veta}
Třída jazyků, které jsou přijímány ZA, odpovídá \emph{bezkontextovým jazykům.}
\end{veta}

\section{Rovnice}
Složitější matematické formulace sázíme mimo plynulý text pomocí prostředí \verb|displaymath|. Lze umístit i~několik výrazů na jeden řádek, ale pak je třeba tyto vhodně oddělit, například příkazem \verb|\quad|. 
$$
1^{{2}^{3}} \neq \Delta^{1}_{{\Delta^{2}}_{\Delta^{3}}}
\quad
y^{11}_{22} - \sqrt[9]{x+{\sqrt[7]{y}}}
\quad
x > y_1 \leq y^{2}
$$
V~rovnici~\eqref{rov2} jsou využity tři typy závorek s~různou \emph{explicitně} definovanou velikostí. Také nepřehlédněte, že nasledující tři rovnice mají zarovnaná rovnítka, a~použijte k~tomuto účelu vhodné prostředí. 
\begin{eqnarray}
- \cos^{2}\beta &=& \frac{\frac{\frac{1}{x}+\frac{1}{3}}{y}+1000}{\prod\limits _{j=2}^8 q_j}\\
\bigg (\Big \{ b \star \big[ 3 \div 4 \big] \circ a \Big\}^{\frac{2}{3}} \bigg) &=& \log_{10} x \label{rov2}\\
\int_a^b f(x)\,\mathrm{d}x &=& \int_c^d f(y)\,\mathrm{d}y \label{rov3}
\end{eqnarray}
V~této větě vidíme, jak vypadá implicitní vysázení limity $\lim _{m \rightarrow \infty} f(m)$ v~normálním odstavci textu. Podobně je to i~s~dalšími symboly jako $\bigcup_{N \in \mathcal{M}} N$ či $\sum_{i=1}^{m} x_{i}^{2}$. S~vynucením méně úsporné sazby příkazem \verb|\limits| budou vzorce vysázeny v~podobě $\lim\limits_{m \rightarrow \infty} f(m)$ a~$\sum\limits^m_{i=1} x^4_i$.

\section{Matice}
Pro sázení matic se velmi často používá prostředí \verb|array| a~závorky (\verb|\left|, \verb|\right|). 
$$
\mathbf{B}=\left|\begin{array}{cccc}
b_{11} & b_{12} & \ldots & b_{1 n} \\
b_{21} & b_{22} & \ldots & b_{2 n} \\
\vdots & \vdots & \ddots & \vdots \\
b_{m 1} & b_{m 2} & \ldots & b_{m n}
\end{array}\right|=\left|\begin{array}{cc}
t & u \\
v & w
\end{array}\right|=t w-u v
$$
$$
\mathbb{X} = \mathbf{Y} \Longleftrightarrow 
\left[
\begin{array}{ccc}
& \Omega + \Delta & \hat{\psi} \\
\vec{\pi}& \omega &
\end{array}
\right] \neq 42
$$

Prostředí \verb|array| lze úspěšně využít i~jinde, například na pravé straně následující rovnice. Kombinační číslo na levé straně vysázejte pomocí příkazu \verb|\binom|.
$$
\binom{n}{k}=\left\{\begin{array}{cl}
0 & \text {pro } k<0\\
\frac{n !}{k !(n-k) !} & \text {pro } 0 \leq k \leq n \\
0 & \text {pro } k>0
\end{array}\right.
$$

\end{document}